\lesson{5}{monday 15 nov 2022 15:15}{Relation database desing by ER and EER}

\section{Mapping of ER- and EERR to relational database}
What do we want to achive with the mapping?

\begin{itemize}
	\item Preserve information, attributes, map them correctly 
	\item maintain constraints, cardinality 1:N etc
	\item Minimize the use of Null values
\end{itemize}

\subsection*{Mapping algorithm}
\begin{itemize}
	\item \textbf{Step 1}
\end{itemize}
In most cases enititys becomes tables. 
For each strong/regular enitity E in the ER schema, create a relation R that includes all \textcolor{red}{simple(?)} attributes of E. Choose on of the key attributes as primary key for R. If this key is a composite, the set of simple attriubutes will together form the key.

\textcolor{red}{We cant have multi value attributes} but we can change it to a new entity. 

\textcolor{red}{read about strong and weak relationships} a weak relationships is weak if the key is dependent on an entity above the relationshio, double line relationship.



\begin{itemize}
	\item \textbf{Step 2}
\end{itemize}
For each weak entitity W in the ER schema with a owner entity E, create a Relation R and include all simple attributes. Also include as foreign key the attributes of R the primary key attributes of the relations(s) that correspond to owner enitity E type(s) \textcolor{red}{QUE?}  

\begin{itemize}
	\item \textbf{Step 3}
\end{itemize}
For each binary 1:1 relationship type R in the ER Schema we should identify the relations S and T that correspond to the entity types that are participating in R. Here there is three different approaches: 

\begin{enumerate}
	\item Foreign key (2 relations) approach. Choose one of the relations S and include a foreign key in S the primary key of T. \textcolor{red}{QUE?} 
	\item Merged relation (1 relation) option. An alternative mapping of a 1:1 relatiopnship type is possible by merging the two entitys types and the relationship inta a single relation.
	\item Cross reference or relationship relation (3 relations) options. The last alternative is to setup a third relation R for the purpose of cross referencing the primary keys of the two relations S and T 
\end{enumerate}

\begin{itemize}
	\item \textbf{Step 4}
\end{itemize}
Mapping of the Binary 1:n relationships types. For each of regular binary 1:N relationships identify the relation S that represent the participating entitiy type at the N side of the relationship. Include as foriegn key in S the primary ket of the relation T that represent the other entity participating in R. Include any simple attributes of the 1:N relation type as attributes of S.

\begin{itemize}
	\item \textbf{Step 5}
\end{itemize}
Mapping of binary M:N relationships. For each binary M:N relationship R, create a new relation S to represent R, this is a \textbf{relationship relation}. Include as a foreign key attributes in S the primary keys of the relations that represent the participating enitity types. \textbf{their combination will form the primary key of S}. Also invlude any simple attributes of the M:N relatioship type as attributes of S.

\begin{itemize}
	\item \textbf{Step 6}
\end{itemize}
Mapping multivalued attributes. For each multivalued attribute A, create a new relation R. This relation will include an attribute corresponding to A, plus the primary key of attribute K-as a foreign key in R of the relation that represent the entitity type of relationship type that A has as an attribute. The primary key of R is the combination of A and K. If this multivalued attribute is composite we incluide its simple components. 

\begin{itemize}
	\item \textbf{Step 7}
\end{itemize}
Mapping of N-ary relationship types. For each n-ary \textcolor{red}{QUE?} relationship type R where n $>$2, we create a new relationship S to represent R. We include R as foreign key attributes in S the pirmary keys of the attributes of the n-ary relationship type or simple components of composite attributes as attributes of S \textcolor{red}{again, QUE?} 

\section{Mapping of generalization and specialization hierarchies}
Mapping EER model constructs to realtions

\begin{itemize}
	\item \textbf{Step 8}
\end{itemize}
Some options for mapping specialization or generalization. Convert each specialization with m subclasses $\{ S_1, S_2, \ldots, S_m \}$ and generlized superclass C, where te attributes of C are $\{ k, a_1, a_2, \ldots, a_n \}$, (where k is the primary key) into relationalm schemas using on of these options

\begin{enumerate}
 	\item Multiple relation-Superclass relations only. \\


 	\item Multiple relations-Subclass relations only \\

 	
 	\item Single relation with one type attribute \\


 	\item Single relation with multiple type attributes. \\


 \end{enumerate} 



\subsection*{Normalization}
\subsection*{Guideline 1}
Information is stored redundantly: 
We should not waste space with having redundant information in the relation table, will making it hard do delete change values or insert new information. 

\subsection*{Functional depencincies}
Are used to specify formal measrues of the "goodness" of relational designs. keys are used to define noermal forms for relations. thet are contraints that are derived from the meaning and \textcolor{red}{internrelationships check}. 

example of FD constraints. SSN determines employee name. Project number determines project name and location. Employee ssn and project number determine the number of hours of the week the employee works with the project. .

In order to understand FDs, we need to understand the meaining of the attriubtes. We need to discuss with the costumer to understand attributes and relationships involved. 

1NF we remove repeated information.

2NF remove partial dependencies. 

3NF trying to elimnate trancietive depencecies. 





