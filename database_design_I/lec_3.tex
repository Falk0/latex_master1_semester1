\lesson{3}{monday 07 nov 2022 15:15}{DBMS and RE models}

\section{ER modelling, recap COMPANY example}

\begin{itemize}
	\item Tips, use colors to identify entities, relationships and attributes. Solutions are not unique and there could be many solutions. Naming scheme, nouns for entities, verbs for relationships. Capital from entitles and relationship. Lower case for attributes. 
	\item Conceptual models are abstract and lacks details but are more formal than natural language. 
	\item Double oval multi-value attribute. 
	\item Trick for many to many, think of table how they are connected. John can only be a manager for one department. The department can have many employees, John, Joseph and Jenny. 
	\item In relationship, 1:1 1:N or N:M. Double it should be a connection, single line is optional. An department may not have any employees but all employees should belong to a department. 
	\item Key should be unique in entity set. Key is underlined. Can be a composite of two attributes. The combination should then be unique. For an example a project can have the same name but the combination with a project number should be unique. 
	\item Weak entity vs strong: weak can not exist without the existence of another strong entity, does not have a key. Dependent is a weak entity, need Employee that is strong and have key. 
	\item Read the ER diagram from top left to bottom right and try to model it this way. 
	\item A way to choose a key is to look at the size of the key, if one is smaller this one could be better to use. 
	\item Relation is a table, be careful with the 
\end{itemize}

\subsection*{Attributes and value sets}
Domains is a range of values for the attribute. 

%Make a snippet for colored text


\subsection*{Relationship of higher degree}
\begin{itemize}
	\item degree 2 called binary
	\item degree 3 called  ternary
\end{itemize}
Avoid using relationship degree higher than two (binary). Constraint are harder to specify for higher (>2) degree relationships. 

\section{Exercise 1}
Galleries  keep  information  about  \textcolor{red}{artists},  their  names  (which  are  unique),  birthplaces,  age, 
and style of art. For each piece of artists, the artist, the year it was made, its unique title, its 
type  of  art  (e.g.,  painting,  lithograph,  sculpture, photograph),  and  its  price  must  be  stored. 
Pieces of \textcolor{red}{artwork} are also \textcolor{blue}{classified/belongs} into \textcolor{red}{groups} of various kinds, for example, portraits, still 
lifes, works by Picasso, or works of the 19th century; a given piece may belong to more than 
one group. Each group is identified by a name (like those just given) that describes the group. 
Finally,  galleries  keep  information  about  customers.  For  each  \textcolor{red}{customer},  galleries  keep  that 
person’s unique name, address, total amount of dollars spent in the gallery (very important!), 
and the artists and groups of art that the customer tends to like. 

\textcolor{red}{Entities}

\textcolor{blue}{Relationship}










