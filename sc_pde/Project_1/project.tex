\documentclass[a4paper]{article}

\usepackage[utf8]{inputenc}
\usepackage[T1]{fontenc}
\usepackage{textcomp}
\usepackage[dutch]{babel}
\usepackage{amsmath, amssymb}


% figure support
\usepackage{import}
\usepackage{xifthen}
\pdfminorversion=7
\usepackage{pdfpages}
\usepackage{transparent}
\newcommand{\incfig}[1]{%
	\def\svgwidth{\columnwidth}
	\import{./figures/}{#1.pdf_tex}
}

\pdfsuppresswarningpagegroup=1
\title{Project 1}
\author{LF CH}
\begin{document}
\maketitle


\begin{equation}
	\begin{aligned}
	u_{tt} = c^2D_2u \quad\quad t \ge 0,\\
	Lw = 0 \quad\quad t \ge 0, \\
	u = f, \quad u_t=0 \quad t=0,
	\end{aligned}
\end{equation}

\begin{equation}
	\begin{aligned}
		u^{T}Hu_{tt}=c^2u^{T}HPD_2Pu = c^2P^{T}HPD_2PU=c^2(Pu)^{T}HD_2Pu= \\
c^2(Pu)^{T}H H^{-1}(-M+BD)Pu = c^2(Pu)^{T}(-M+BD)Pu =\\
-c^2(Pu)^{T}MPu + c^2(Pu)^{T}BDPu \\
+ u^{T}_{tt}Hu = -c^2(Pu)^{T}M^{T}Pu + c^2(Pu)^{T}BD^{T}pu\\ \hline
= u^{T}Hu_{t t} + u^{T}_{t t}Hu = -c^2(Pu)^{T}(M+M^{T})Pu + c^2(Pu)^{T}(BD + BD^{T})Pu \\
= \frac{d}{dt} \|u_t\|^{2}_H = -2c^{2}(Pu)^{T}MPu + c^2(Pu)^2BDPu + c^2BD^{T}Pu
	\end{aligned}
\nonumber
\end{equation}

$B = e_md_m-e_1d_1$ and let $w = PV$. 

\begin{equation}
\begin{aligned}
	\frac{d}{dt}\|v_t\|^2_H = -2c^2w^{T}Mw +c^2w^{T}(e_md_m-e_1d_1)w + c^2w^{T}(e_md^{T}_m - e_1d^{T}_1)w = \\
	-2c^2w^{T}Mw + c^2w^{T}e_md_mw - c^2w^{T}e_1d_1w + c^2w^{T}e_md^{T}_mw - c^2w^{T}e_1d^{T}_1w 
\end{aligned}
\nonumber
\end{equation}

\section{The system}
When introducing $u_t$ to the boundary operator we cant use the previous form of semi-discrete approximation. We therefore introducing the variable substitution $v = u_t$. By creating a system of first order equations we can solve it with RK4. 

\begin{equation}
	\begin{aligned}
		\begin{bmatrix} u \\ v \end{bmatrix}_t = \begin{bmatrix} 0& I \\ c^2D_2& 0 \end{bmatrix}\begin{bmatrix} u \\ v \end{bmatrix} \\
		L\begin{bmatrix} u \\ v \end{bmatrix} = 0 \\
		\begin{bmatrix} u \\ v \end{bmatrix} = \begin{bmatrix} f \\ 0 \end{bmatrix} 
	\end{aligned}
\end{equation}

We introduce $e^{(1)}$, $e^{(2)}$ and the Kronecker-operator to form the boundary operator

\begin{equation}
	\begin{aligned}
		e^{(1)}=\begin{bmatrix} 1 \\ 0 \end{bmatrix}, e^{(2)}=\begin{bmatrix} 0 \\ 1 \end{bmatrix} 
	\end{aligned}
\end{equation}

\begin{equation}
	\begin{aligned}
		L \begin{bmatrix} u \\ v \end{bmatrix}  = \begin{bmatrix} \alpha_l(e^{(2)} \otimes e_1 ) + \beta_l(e^{(1)} \otimes e_1 ) + \gamma_li(e^{(1)} \otimes d_1 ) \\ \alpha_r(e^{(2)}\otimes e_m) + \beta_r(e^{(1)} \otimes e_m ) + \gamma_r (e^{(1)} \otimes d_m) \end{bmatrix} 
	\end{aligned}
\end{equation}

Apply the projection and  energy method to verify stability. We multiply with $\begin{bmatrix} u \\ v \end{bmatrix}^{T}\overline{H} $ and let: $\overline{H} = \begin{bmatrix} 1& 0 \\ 0& 1 \end{bmatrix}H $

\begin{equation}
	\begin{aligned}
		\begin{bmatrix} u \\ v \end{bmatrix}^{T}\overline{H}  \begin{bmatrix} u \\ v \end{bmatrix}_t = \begin{bmatrix} u \\ v \end{bmatrix}^{T}\overline{H}  P \begin{bmatrix} 0& 1 \\c^2D_1& 0 \end{bmatrix}P \begin{bmatrix} u \\ v \end{bmatrix}  = \left(P\begin{bmatrix} u \\v \end{bmatrix}\right) ^{T} \overline{H} \begin{bmatrix} 0& 1 \\ c^2D_2& 0 \end{bmatrix} P \begin{bmatrix} u \\ v \end{bmatrix} = \\ 
		\left( P\begin{bmatrix} u \\v \end{bmatrix} \right)^{T} \begin{bmatrix} 0& H \\ c^2D_2H& 0 \end{bmatrix} P \begin{bmatrix} u \\ v \end{bmatrix} =  	\left( P\begin{bmatrix} u \\v \end{bmatrix} \right)^{T} \begin{bmatrix} 0& H \\ c^2HH^{-1}(-M+BD)& 0 \end{bmatrix} P \begin{bmatrix} u \\ v \end{bmatrix}  \\ 
		= \left( P\begin{bmatrix} u \\v \end{bmatrix} \right)^{T} \begin{bmatrix} 0& H \\ c^2(-M+BD)& 0 \end{bmatrix} P \begin{bmatrix} u \\ v \end{bmatrix} \\
+  \left( P\begin{bmatrix} u \\v \end{bmatrix} \right)^{T} \begin{bmatrix} 0& H \\ c^2(-M+BD^T)& 0 \end{bmatrix} P \begin{bmatrix} u \\ v \end{bmatrix} \\ \hline
\frac{d}{dt} \|\begin{bmatrix} u \\ v \end{bmatrix} \|_H =  \left( P\begin{bmatrix} u \\v \end{bmatrix} \right)^{T} \begin{bmatrix} 0& H \\ c^2(-2M - BD - BD^T)& 0 \end{bmatrix} P \begin{bmatrix} u \\ v \end{bmatrix} = \\
PHuPv + c^2Pv(-2M-BD-BD^{T})Pu =\\
PHuPv - c^2Pv 2Mpu + Pc^2vBDPu + pc^2vBDPu = \\
PHuPv - c^2Pv 2MPu + Pc^2v(e_md_m-e_1d_1)Pu + c^2Pv(e_md^{T}_m-e_1d^{T}_1)Pu= \\
PHuPv - c^2Pv 2MPu + Pc^2ve_md_mPu -  Pc^2ve_1d_1Pu + Pc^2ve_md^{T}_mPu - Pc^2ve_1d^{T}_1Pu
\end{aligned} 2	
 \nonumber
\end{equation}

Using that $d_1u = \frac{1}{c}e^{T}_1v $ and $d_m = -\frac{1}{c} e^{T}_mv$


\begin{equation}
	\begin{aligned}
	PHuvP - c^2Pv 2MuP + cPve_m e^{T}_mv P -  cPve_1 e^{T}_1vP + c^2Pve_md^{T}_muP - c^2Pve_1d^{T}_1uP = \\
	\end{aligned}
	\nonumber
\end{equation}

\end{document}
