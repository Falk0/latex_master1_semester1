›\lesson{4}{wednesday 09 nov 2022 15:15}{Filtering part II}

\section{Image filtering in frequency domain}

summary, Virtually all filtering is local neighborhood operation.
Convolution = linear and shift invariant filters, mean filter Gaussian weighted filter , kernel. 
Many non linear filter exist also. 

\subsection*{Linear neighborhood operation}
For each pixel multiply the values in the neighborhood with the corresponding weights then sum. Is convolution as long at is symmetric. When not its a cross correlation. 

a filter is symmetric if we flip it along x or y axis. 

\subsection*{Correlation and convolution}
 \begin{itemize}
 	\item Two fundamental linear filtering operation
 	\item Correlation: move a filter mask compute and sum
 	\item  Convolution the same as correlation but first rotate filter by 180 degrees.  
 \end{itemize}
 
 \begin{example}{Example: correlation}
 Signal: \\
 0 0 0 0 0 0 1 0 0 0 0 0 0 \\
 Filter: \\
 321 \\
 Result Correlation: \\
 0 0 0 0 1 2 3 0 0 0 0 \\
 Result Convolution: \\
 0 0 0 0 3 2 1 0 0 0 0 \\
 \begin{itemize}
 	\item Convolving a function with a unit impulse yields a copy. Correlation we flip it. 
 \end{itemize}
 
 \end{example}	



\subsection*{Convolution properties:}

\begin{itemize}
	\item Linear
	\begin{itemize}
		\item Scaling invariant 
		\item Distributive 
	\end{itemize}
	\item time invariant
	\item Commutative 
	\item Associative
\end{itemize}

\subsection*{Today's lecture:}

\begin{itemize}
	\item The Fourier transform
	\begin{itemize}
		\item Discrete Fourier Transform 
		\item Fourier transform in 2D
		\item Fast Fourier Transform
	\end{itemize}
	\item signing filters in Fourier domain
	\begin{itemize}
		\item Filtering out structured noise
	\end{itemize}
	\item Sampling aliasing and interpolation. 
\end{itemize}

All periodic functions can be expressed as weighted sum of trigonometric functions. Even functions that are not periodic van be expressed as an integral of sines and cosines. 

\textbf{Euler's formula:}

\begin{equation}
e^{i\delta} = \text{cos}\omega\delta + i sin \omega\delta, \quad \omega = 2\pi \text{f}
\end{equation}

\textbf{Fourier transform }
\begin{equation}
F(\omega) = \int_{-\infty}^{\infty} \ f(x) e^{i\omega x},\text{dx}
\end{equation}

\begin{equation}
Ae^{i\omega x} + A^{*}e^{-i\omega x}
\end{equation}

for real valued signal;:
at frequency $\omega$ we have weighted A \textcolor{red}{missing text... continue when slides uploaded}  %Continue here 

We can go back with the inverse Fourier transform

\begin{equation}
f(x) = \frac{1} {2\pi} \int_{-\infty}^{\infty} F(\omega) e^{ix} \,\text{d} 
\end{equation}

\textcolor{red}{Add Image of of transform pairs here ....}  

Different properties: scaling, addition translation and convolution. Convolution in frequency domain is multiplication in frequency domain, very useful.. 


Sampling, Continuous function, sampling function and sampled function example. 

Examples of DFT

DFT
\begin{equation}
F[k] = \sum_{n=0}^{N-1} f[n] e^{-i \frac{2\pi} {N}kn } %change for bracket
\end{equation}

k is the spatial frequency.

\subsection*{Discrete Fourier Transform}
F[k] is defined on a limited domain (N samples) these samples are assumed to repeat periodically



\textbf{Question: Why does the DFT only have positive freq?}\\
Its periodic, we store one copy of it because it symmetric  

\textbf{Question: what is the zero frequency?} 


\begin{equation}
F(k) = \sum_{n=0}^{N-1} f(n)e^{-i \frac{2\pi} {N}2n } % check this
\end{equation}
The exponential cancel out, become 1. and if we sum it and divide it by N we get the average.


\section{Fourier transform in 2D}
Simple, the FT is separable
\begin{itemize}
	\item perform transform x-axis
	\item perform transform along y-axis
	\item perform ...
\end{itemize}

2D transform pairs examples: 

Taking the transform result in same size but the result is complex so we look at the magnitude. The middle is the average. The Fourier transform gives us a direct hint of the image. The frequency content in different axis.

Sine like pattern gives dots in the Fourier transform, showing the frequency clearly. 

We can also visualize the angle/phase. But doesn't intuitively say much. Reconstructing from only magnitude doesn't give much, the phase give some but both needed for reconstruct. 

\subsection*{Computing the DFT}
\begin{itemize}
	\item for an image with N pixels the DFT contains N elements
	\item each element of the DFT can be m \textcolor{red}{finish when slides are uploaded} 
	\item a naive approach need $N^{2}$
\end{itemize}


\subsection*{Fast Fourier Transform }
\begin{itemize}
	\item clever algorithm to compute Fourier transform
	\item runs in O(NlogN) time
	\item works because of symmetry properties
\end{itemize}

\subsection*{Convolution in Fourier domain}
The convolution prop in freq domain. This we can calculate the convolution through

\begin{itemize}
	\item F = FFT(f)
	\item H = FFT(h)
	\item G = G $\times$ H
	\item g = IFFT(G)
\end{itemize}

Convolution is operation of O(MN) Through the FFT O(NlogN) if M larger than NlogN then its of more practical use. Use depend on size of filter. 

We don't lose information going back and forth the freq domain. If no numerical/double/float error. 

\subsection*{Low pass filtering}
Linear smoothing filter are all low pass filter. Mean filter and Gauss filter. Low pass means low freq are not altered and high are attenuated. In the ideal case. 

\subsection*{Highpass filtering}
Opposite of lowpass, the unsharp mask and Laplace are highpass filter

\subsection*{Bandpass filter}
You can chose to keep one passband of frequencies to keep and filter the other.

Rippling pillbox gives ringing. Image example.

Gaussian filter has much smoother properties and doesn't give the same kind of ringing. 

\section{Fourier analysis of sampling}
\begin{equation}
F(\omega) = \int_{-\infty}^{\infty} f(x) e^{-i\omega t}\,\text{dx}
\end{equation} %check

If we get a repeated copy we can reconstruct the sampled  signal. But if we get aliasing, colliding repeating part we cant. This is caused by not sampling in high enough freq. 

We cut out the repeating part with a box function in frequency domain(sinc in spatial/time domain) 

Nyquist theorem:  if you want to reconstruct a signal you need to sample 2 times higher than the highest frequency content. 

How do we know that the captured image is band limited?




