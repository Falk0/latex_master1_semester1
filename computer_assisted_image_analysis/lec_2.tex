\documentclass[a4paper]{article}

\usepackage[utf8]{inputenc}
\usepackage[T1]{fontenc}
\usepackage{textcomp}
\usepackage[english]{babel}
\usepackage{amsmath, amssymb}
\usepackage{natbib}
\usepackage{float} 
\usepackage[caption = false]{subfig}
\usepackage{listings}
\lstset{
    breaklines=true,
    basicstyle=\tt\normalsize,
    keywordstyle=\color{blue},
    identifierstyle=\color{magenta},
    frame = single
} 
% figure support
\usepackage{import}
\usepackage{xifthen}
\pdfminorversion=7
\usepackage{pdfpages}
\usepackage{transparent}
\pdfsuppresswarningpagegroup=1

\title{Image analysis \\ lecture 2}
\author{Linus Falk}
\begin{document}
\maketitle
\section{Image enhancment}
Enhance part of an image in some way. Transform an image into a new image. Create a better. resore information, reduce noice.
Enhance certain details, edges etc. Just look better. 
We dont increase the information. Can be performed in spatial domain. Point process : pixel base. Filter works with neighborhood
Another way is in frequency domain. 
The chain of image analysis process. The first part today. Preprocessing enhancment. We must start by understanding the problem otherwise it hard to make decisions along the chain/pipeline


\begin{itemize}
    \item image arithemic
    \item intensity transfer function
    \item histogram and histogram equalization
\end{itemize}


\subsection{image arit´metics}
we do aritmetics with images. Postion in matrix/image operator position in image. 

\begin{itemize}
    \item standar operation + - / *
    \item logic operator AND OR XOR
\end{itemize}

Pitfalls could be add and divide could be outside range of 0-255 for example. Need to normalize but needs to be done before we do the operation. otherwise we might have destroyed information. Bit depth important. 

Useful way is to truncate the image. 

We can subtract images. Leaf example and chessboard example. Binary or greyscale images. 

Atitmetics usefull when parts of images should be excluded for example.

Logical operator in binary images, pixel example in slides. Nothing strange. But good method to add or remove certain objects in binary images. 

\subsection{noise reduction}
using mean or median, useful in microscopy and night pictures/astrionomy

\begin{equation}
I = \frac{1} {N} \sum_{n=1 \ldotsn }I_k
\end{equation}

Reduction of noise by using the mean of the pixels by using images of the same scene. Good in microscopy and astronomy were the scene doesn't change. 

\subsection{application}

\begin{itemize}
    \item Image arithmetic useful in medication/diagnosis. Subtracting picture before contrast fluid with the picture after to get en enhancement/ better picture of the blood vessels. 
    \item change or motion in a scene. Persons in scene before and after example. 
    \item illumination correction by subtraction background image. Max or median of the pixel intensities. 
\end{itemize}


\section{intensity transfer function}

\begin{equation}
g(x,y) = Tf(x,y)
\end{equation}

\begin{itemize}
    \item linear (neutral negative, contrast, brightness)
    \item smooth, gamma log
    \item arbitrary
\end{itemize}

old value on x axis new on y axis.

\subsection{The negative transformation}
the inverse 
\begin{equation}
g(x,y) = max - f(x,y)
\end{equation}

\begin{equation}
\begin{aligned}
\begin{bmatrix} 255& 254& 253 \\ 
                125& 130& 110 \\ 
                4& 3& 0 \end{bmatrix} 
                \Rightarrow 
\begin{bmatrix} 0& 1& 2 \\ 
        130& 125& 145 \\ 
        251& 252& 255
        \end{bmatrix}
\end{aligned}
\end{equation}

useful in medical image processing. Retina example. Easier to distinguish brighter lines/object. Sometimes the opposite.

\subsection{Brightness}
If we add a constant to the image it becomes brighter. Subtracting will make it darker. C positive integer or

\begin{equation}
g(x,y) = f(x,y) + C
\end{equation}


\subsection{Contrast}
By multiplying the image we spread out the information and increases the contrast

\begin{equation}
g(x,y) = f(x,y) \times C, C > 1
\end{equation}

if C < 1 reduce the contrast. 


\subsection{Gamma transformation}

\begin{equation}
g(x,y) = C \times f(x,y)^{\gamma}
\end{equation}

Computer monitors $ \gamma \approx 2.2$

\begin{itemize}
    \item Computer monitors $ \gamma \approx 2.2$
    \item eyes $\approx 0.45$
    \item microscopes $\approx 1$
\end{itemize}

microscopes should have 1. 1 to 1 ratio. Lower gamma brighter image. Gamma high, darker. 

\subsection{Log transformation}
Used to visualize dark regions of an image. To display the fourier spectrum. Enhance the brighter regions. 

\subsection{arbitrary}
only one output per input. Possibly not continuous.


\section{Intensity histogram and histogram equalization}



\begin{equation}
g(x,y) = Clog(1 + f(x,y))
\end{equation}

\bibliographystyle{unsrt}
\bibliography{references}
\end{document}