\documentclass[a4paper]{article}

\usepackage[utf8]{inputenc}
\usepackage[T1]{fontenc}
\usepackage{textcomp}
\usepackage[english]{babel}
\usepackage{amsmath, amssymb}
\usepackage{natbib}
\usepackage{float} 
\usepackage[caption = false]{subfig}
\usepackage{listings}
\lstset{
    breaklines=true,
    basicstyle=\tt\normalsize,
    keywordstyle=\color{blue},
    identifierstyle=\color{magenta},
    frame = single
} 
% figure support
\usepackage{import}
\usepackage{xifthen}
\pdfminorversion=7
\usepackage{pdfpages}
\usepackage{transparent}
\pdfsuppresswarningpagegroup=1

\title{Image analysis \\ lecture 2}
\author{Linus Falk}
\begin{document}
\maketitle
\section{Image enhancment}
Enhance part of an image in some way. Transform an image into a new image. Create a better. resore information, reduce noice.
Enhance certain details, edges etc. Just look better. 
We dont increase the information. Can be performed in spatial domain. Point process : pixel base. Filter works with neighborhood
Another way is in frequency domain. 
The chain of image analysis process. The first part today. Preprocessing enhancment. We must start by understanding the problem otherwise it hard to make decisions along the chain/pipeline


\begin{itemize}
    \item image arithemic
    \item intensity transfer function
    \item histogram and histogram equalization
\end{itemize}


\bibliographystyle{unsrt}
\bibliography{references}
\end{document}