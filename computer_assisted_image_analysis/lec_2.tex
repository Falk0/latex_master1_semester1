›\lesson{2}{tuesday 01 nov 2022 10:15}{Image arithmetic}
\section{Image enhancement}
Enhance part of an image in some way. Transform an image into a new image. Create a better. restore information, reduce noise.
Enhance certain details, edges etc. Just look better. 
We don't increase the information. Can be performed in spatial domain. Point process : pixel base. Filter works with neighborhood
Another way is in frequency domain. 
The chain of image analysis process. The first part today. Pre processing enhancement. We must start by understanding the problem otherwise it hard to make decisions along the chain/pipeline


\begin{itemize}
    \item image arithmetic
    \item intensity transfer function
    \item histogram and histogram equalization
\end{itemize}


\subsection*{image arithmetic}
we do arithmetic with images. Position in matrix/image operator position in image. 

\begin{itemize}
    \item standard operation + - / *
    \item logic operator AND OR XOR
\end{itemize}

Pitfalls could be add and divide could be outside range of 0-255 for example. Need to normalize but needs to be done before we do the operation. otherwise we might have destroyed information. Bit depth important. 

Useful way is to truncate the image. 

We can subtract images. Leaf example and chessboard example. Binary or grey scale images. 

Arithmetic useful when parts of images should be excluded for example.

Logical operator in binary images, pixel example in slides. Nothing strange. But good method to add or remove certain objects in binary images. 

\subsection*{noise reduction}
using mean or median, useful in microscopy and night pictures/astronomy

\begin{equation}
I =  \frac{1} {N} \sum_{n=1 \ldots n }I_k
\end{equation}

Reduction of noise by using the mean of the pixels by using images of the same scene. Good in microscopy and astronomy were the scene doesn't change. 

\subsection*{application}

\begin{itemize}
    \item Image arithmetic useful in medication/diagnosis. Subtracting picture before contrast fluid with the picture after to get en enhancement/ better picture of the blood vessels. 
    \item change or motion in a scene. Persons in scene before and after example. 
    \item illumination correction by subtraction background image. Max or median of the pixel intensities. 
\end{itemize}


\section{intensity transfer function}

\begin{equation}
g(x,y) = Tf(x,y)
\end{equation}

\begin{itemize}
    \item linear (neutral negative, contrast, brightness)
    \item smooth, gamma log
    \item arbitrary
\end{itemize}

old value on x axis new on y axis.

\subsection*{The negative transformation}
the inverse 
\begin{equation}
g(x,y) = max - f(x,y)
\end{equation}


\begin{example}{An example}

\begin{equation}
\begin{aligned}
\begin{bmatrix} 255& 254& 253 \\ 
                125& 130& 110 \\ 
                4& 3& 0 \end{bmatrix} 
                \Rightarrow 
\begin{bmatrix} 0& 1& 2 \\ 
        130& 125& 145 \\ 
        251& 252& 255
        \end{bmatrix}
\end{aligned}
\end{equation}

\end{example}

Useful in medical image processing. Retina example. Easier to distinguish brighter lines/object. Sometimes the opposite.

\subsection*{Brightness}
If we add a constant to the image it becomes brighter. Subtracting will make it darker. C positive integer or

\begin{equation}
g(x,y) = f(x,y) + C
\end{equation}


\subsection*{Contrast}
By multiplying the image we spread out the information and increases the contrast

\begin{equation}
g(x,y) = f(x,y) \times C, C > 1
\end{equation}

if C < 1 reduce the contrast. 


\subsection*{Gamma transformation}

\begin{equation}
g(x,y) = C \times f(x,y)^{\gamma}
\end{equation}

Computer monitors $ \gamma \approx 2.2$

\begin{itemize}
    \item Computer monitors $ \gamma \approx 2.2$
    \item eyes $\approx 0.45$
    \item microscopes $\approx 1$
\end{itemize}

microscopes should have 1. 1 to 1 ratio. Lower gamma brighter image. Gamma high, darker. 

\subsection*{Log transformation}
Used to visualize dark regions of an image. To display the Fourier spectrum. Enhance the brighter regions. 

\begin{equation}
g(x,y) = Clog(1 + f(x,y))
\end{equation}

\subsection*{arbitrary}
only one output per input. Possibly not continuous.


\section{Intensity histogram and histogram equalization}

Gray scale histogram show how many pixels at each intensity level. 

Normalized histogram: normalized by the total number of pixels in the image. Histogram show intensity distribution. How many pixels of certain intensity.

Intensity histogram doesn't say anything about the spatial distribution of pixel intensities. Images with the same pixels histogram can  be totally different.

What do we use them for? 
\begin{itemize}
    \item Thresholding, intensity threshold. Decide intensity all above or under is background. Works with bi-modal histogram
    \item analyze the brightness and contrast
    \item histogram equalization
\end{itemize}

Analyze the brightness. See the transformation "chopping" the histogram. Could see that information might be missing. Low contrast = compressed histogram. When increasing we stretch the histogram. Transfer function slope. 

\subsection*{histogram equalization}
create an histogram with evenly distribution grey levels. for visual contrast enhancement. The goal is to flatten the histogram, produce the most even histogram. 

\subsection*{Cumulative histogram}
sum the number of pixels along the x axis intensity.
Steep slope. Intensely populated parts of the histogram. Flat slope: sparsely populated parts of the histogram. Strive to a even slope. 

\subsection*{example CDF}
Multiplying the CDF value with  number of gray levels -1 gives the intensity transfer function. We can the map the new gray level values into the number of pixels. it possible that two bins will be mapped to the same new position. 

\textcolor{red}{look at this again}

\subsection*{local histogram equalization}
useful when only parts of image need to be enhanced 

\subsection*{Conclusion Image arithmetic}
\begin{itemize}
    \item useful when histogram narrow
    \item drawback, amplifies noise, can produce unrealistic transformation
    \item information can be lost. no new information gained. 
    \item Not invertible, usually destructive. 
\end{itemize}

Usefulness depends on the amount of different intensities. 

\subsection*{Histogram matching}
Want to mimic histogram of another image. Compute the histograms and CDF for each image. For each gray level $G_1 \begin{bmatrix} 0& 255 \end{bmatrix}$ find graylevel $G_2$ so $F_1(G_1) = F_2(G_2)$
The matching function: $M(G_1) = G_2$. Not always the best solution either. 



\section{Summary Image arithmetic}

\begin{itemize}
    \item Many common tasks can be described by image arithmetic
    \item histogram eq useful for visualization 
    \item watch out for information leaks
\end{itemize}

to think about
\begin{itemize}
    \item relation between arithmetic and linear transfer function
    \item what can we know of an image from the histogram
    \item 8-bit image A, how will it look like B = 255*(A+1)
    \item conclusion if first last column really high?
    \item better resolution combining multiple images of same sample?
\end{itemize}

